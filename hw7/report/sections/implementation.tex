\section{Implementation Details}
\label{sec:implementation}

\subsection{Task1: A2C in \pendulum}
\label{sec:task1}

\subsubsection{Stochastic Policy Gradient Calculation}

In the \texttt{update\_model()} method, the policy gradient is computed as:

\inputminted[autogobble, firstline=179, lastline=212, highlightlines={202-205}]{python}{../a2c_pendulum.py}

This represents the advantage-weighted log probability gradient, augmented with an entropy regularization term to encourage exploration.

\subsubsection{TD Error Estimation}

Temporal Difference (TD) error is computed using the Bellman equation, as shown bellow:

\inputminted[autogobble, firstline=179, lastline=212, highlightlines={191-192,194}]{python}{../a2c_pendulum.py}

This helps the critic estimate future returns using bootstrapped values.

\subsection{Task2: PPO-Clip with GAE on \pendulum}
\label{sec:task2}

\subsubsection{Clipped Objective}

In the \texttt{update\_model()} method of \texttt{PPOAgent} (in \texttt{ppo\_pendulum.py}), we compute the clipped objective as:

\inputminted[autogobble, firstline=236, lastline=301, highlightlines={271,273-278}]{python}{../ppo_pendulum.py}

This clipping mechanism ensures that the policy update does not move too far from the old policy, improving training stability.

\subsubsection{Generalized Advantage Estimator (GAE)}

The GAE is implemented in the \texttt{compute\_gae()} method, which calculates the advantage estimates using a combination of TD errors and bootstrapped values:
\inputminted[autogobble, firstline=96, lastline=104]{python}{../ppo_pendulum.py}

Here, gamma ($\gamma$) and tau ($\tau$) are user-configurable hyperparameters that balance bias and variance in advantage estimation.

\subsection{Task3: PPO with GAE on \walker}
\label{sec:task3}

\subsubsection{Sample Collection from the Environment}

Transitions are collected similarly as in Task 2 (Section \ref{sec:task2}), using the \texttt{select\_action()} and \texttt{step()} methods in \texttt{ppo\_walker.py}.
These transitions are stored in memory and later batched during training with \texttt{ppo\_iter()}.

\subsubsection{Exploration Enforcement}

The entropy bonus is included in the policy loss to ensure exploration:

\inputminted[autogobble, firstline=274, lastline=279, highlightlines=278]{python}{../ppo_walker.py}

he coefficient for this term can be adjusted using the \texttt{----entropy-weight} CLI parameter (see \texttt{main()} function).

\subsection{Model Performance Tracking}

Since we are required to complete 3 different tasks in two different environments, I used the \href{https://wandb.ai/site}{Weight \& Biases} (WandB) library to track the training process and visualize the results.

\subsubsection{Categorizing the Results}

To categorize the results, I used the \texttt{wandb.init} method to create a new run for each task with different Project names.
The reson why not using tags to catogorize the results as I did in homework 5 is that I wanted to have a clear separation between the different tasks and environments.
This allows me to drag different dashboards avoid confusion and easily compare the results of different runs.

\subsubsection{Snapshotting the Code}

I also used set argument \texttt{save\_code} to \texttt{True}.
This allows me to easily reproduce the results and compare the code used for different runs.

\subsubsection{Hyperparameter Tuning}

I used the \texttt{wandb.config} method to define the hyperparameters for each run.
This allows me to easily track and compare the hyperparameters used for different runs in the WandB dashboard.

\subsubsection{Result Plot Generation}

I used the \texttt{wandb.log} method to log the results of each run.
This allows me to easily visualize the results in the WandB dashboard and compare the performance of different runs.
This also allows me to easily export the plots and use them in the report.
