\section{Analysis \& Discussion}

\subsection{Teacher Forcing Ratio}
I experimented with different initial teacher forcing ratios (\texttt{tfr}) and decay strategies. The ratio controls the probability of using the ground truth frame $x_{t-1}$ instead of the generated frame $\hat{x}_{t-1}$ during training. The configurations I tested include:

\begin{itemize}
    \item \textbf{TF1:} \texttt{tfr = 1.0}, \texttt{tfr\_sde = 0}, \texttt{tfr\_d\_step = 0.1}
    \item \textbf{TF2:} \texttt{tfr = 0.5}, \texttt{tfr\_sde = 10}, \texttt{tfr\_d\_step = 0.05}
    \item \textbf{TF3:} \texttt{tfr = 0.0} (no teacher forcing)
\end{itemize}

These experiments were conducted with default parameters, including a batch size of 4 and a learning rate of 0.0001.
The first configuration used a high initial TFR of 1.0, which decayed to 0.0 over 10 epochs.
The second configuration started with a TFR of 0.5 and start decaying after 10 epochs, reaching 0.0 after 20 epochs.
The third configuration used no teacher forcing, meaning the model always used its own predictions as input.

Figure~\ref{fig:tfr_vs_loss} shows the decay of the TFR and its relationship to the training loss. I observed that high initial TFR led to faster convergence early on, but sometimes caused the model to overfit to teacher inputs. In contrast, training with \texttt{tfr = 0.0} converged more slowly initially, but achieved better consistency in frame generation.

\begin{figure}[H]
    \centering
    \begin{tikzpicture}
    \begin{groupplot}[
            group style={
                    group size=1 by 2,
                    vertical sep=2.5cm,
                },
            width=0.98\linewidth,
            height=0.3\linewidth,
            xlabel={Epoch},
            grid=major,
            legend columns=1,
            legend style={
                    anchor=north,
                    column sep=1ex,
                    legend cell align=left,
                },
            legend to name=tfrlegend,
        ]

        % --- Top: TFR vs Loss ---
        \nextgroupplot[
            ylabel={TFR},
            title={Teacher Forcing Ratio (TFR)},
            tick align=outside,
        ]
        \addplot[mark=-,orange] table [
                col sep=comma,
                x=Step,
                y=Value
            ]{../csvs/tfr1_tfr.csv};
        \addlegendentry{TF1: \texttt{tfr = 1.0, tfr\_sde = 0, tfr\_d\_step = 0.1}}
        \addplot[mark=-,violet] table [
                col sep=comma,
                x=Step,
                y=Value
            ]{../csvs/tfr2_tfr.csv};
        \addlegendentry{TF2: \texttt{tfr = 0.5, tfr\_sde = 10, tfr\_d\_step = 0.05}}
        \addplot[mark=-,cyan] table [
                col sep=comma,
                x=Step,
                y=Value
            ]{../csvs/tfr3_tfr.csv};
        \addlegendentry{TF3: \texttt{tfr = 0}}

        % --- Bottom: Loss ---
        \nextgroupplot[
            ylabel={Loss},
            title={Training/Validation Loss},
            height=0.5\linewidth,
            ymode=log,
            ymax=0.02,
            ymin=0.0002,
            tick align=outside,
        ]
        \addplot[mark=-,orange] table [
                col sep=comma,
                x=Step,
                y=Value
            ]{../csvs/tfr1_loss.csv};
        \addlegendentry{TF1: \texttt{tfr = 1.0, tfr\_sde = 0, tfr\_d\_step = 0.1}}
        \addplot[mark=-,violet] table [
                col sep=comma,
                x=Step,
                y=Value
            ]{../csvs/tfr2_loss.csv};
        \addlegendentry{TF2: \texttt{tfr = 0.5, tfr\_sde = 10, tfr\_d\_step = 0.05}}
        \addplot[mark=-,cyan] table [
                col sep=comma,
                x=Step,
                y=Value
            ]{../csvs/tfr3_loss.csv};
        \addlegendentry{TF3: \texttt{tfr = 0}}

    \end{groupplot}
\end{tikzpicture}
\centering
\ref*{tfrlegend}

    \caption{Teacher Forcing Ratio and Corresponding Loss}
    \label{fig:tfr_vs_loss}
\end{figure}

\subsection{Loss Curve Under Different KL Annealing Strategies}

I compared three KL annealing strategies:

\begin{itemize}
    \item \textbf{KL1: Cyclical}: Repeated ramp-up and reset schedule
    \item \textbf{KL2: Monotonic}: Linearly increasing $\beta$
    \item \textbf{KL3: Constant}: No annealing; $\beta = 1$ throughout
    \item \textbf{KL4: Cyclical R0.5}: Cyclical with $\texttt{kl\_anneal\_ratio} = 0.5$
\end{itemize}

These experiments were conducted with a batch size of 4 and a learning rate of $0.0001$.
The parameters for the cyclical schedule were $\texttt{kl\_anneal\_ratio} = 1$ and $\texttt{kl\_anneal\_cycle} = 10$.
And the parameters for cyclical R0.5 schedule were $\texttt{kl\_anneal\_ratio} = 0.5$ and $\texttt{kl\_anneal\_cycle} = 10$.


Figure~\ref{fig:loss_kl_modes} shows the training loss curves. As expected, the \texttt{monotonic} and \texttt{cyclical} schedules provided more stable training in the early epochs. Without KL annealing, the KL term dominated early, leading to unstable gradients and degraded frame quality. Among the three, the \texttt{cyclical} method—based on Fu et al.~\cite{fu_cyclical_2019_naacl}—produced the best overall convergence and avoided KL collapse.

\begin{figure}[H]
    \centering
    \begin{tikzpicture}
    \begin{groupplot}[
            group style={
                    group size=1 by 2,
                    vertical sep=2.5cm,
                },
            width=0.98\linewidth,
            height=0.3\linewidth,
            xlabel={Epoch},
            grid=major,
            legend columns=5,
            legend style={
                    anchor=north,
                    column sep=1ex,
                    legend cell align=left,
                },
            legend to name=kllegend,
        ]

        % --- Top: KL Annealing Schedule ---
        \nextgroupplot[
            ylabel={Beta},
            title={KL Annealing Schedule},
            tick align=outside,
        ]
        \addplot[mark=-,cyan] table [
                col sep=comma,
                x=Step,
                y=Value
            ]{../csvs/kl1_beta.csv};
        \addlegendentry{KL1: Cyclical}
        \addplot[mark=-,orange] table [
                col sep=comma,
                x=Step,
                y=Value
            ]{../csvs/kl2_beta.csv};
        \addlegendentry{KL2: Monotonic}
        \addplot[mark=-,violet] table [
                col sep=comma,
                x=Step,
                y=Value
            ]{../csvs/kl3_beta.csv};
        \addlegendentry{KL3: Constant}
        \addplot[mark=-,purple] table [
                col sep=comma,
                x=Step,
                y=Value
            ]{../csvs/kl4_beta.csv};
        \addlegendentry{KL4: Cyclical R0.5}

        % --- Bottom: Loss ---
        \nextgroupplot[
            ylabel={Loss},
            title={Training/Validation Loss},
            ymode=log,
            ymax=0.1,
            ymin=0.0001,
            tick align=outside,
        ]
        \addplot[mark=-,cyan] table [
                col sep=comma,
                x=Step,
                y=Value
            ]{../csvs/kl1_loss.csv};
        \addlegendentry{KL1: Cyclical}
        \addplot[mark=-,orange] table [
                col sep=comma,
                x=Step,
                y=Value
            ]{../csvs/kl2_loss.csv};
        \addlegendentry{KL2: Monotonic}
        \addplot[mark=-,violet] table [
                col sep=comma,
                x=Step,
                y=Value
            ]{../csvs/kl3_loss.csv};
        \addlegendentry{KL3: Constant}
        \addplot[mark=-,purple] table [
                col sep=comma,
                x=Step,
                y=Value
            ]{../csvs/kl4_loss.csv};
        \addlegendentry{KL4: Cyclical R0.5}
    \end{groupplot}
\end{tikzpicture}
\centering
\ref*{kllegend}

    \caption{Training Loss with Different KL Annealing Strategies}
    \label{fig:loss_kl_modes}
\end{figure}

\subsection{PSNR-per-Frame in Validation Set}

To evaluate the quality of the generated frames, I computed the PSNR between each predicted frame and its ground truth counterpart on the validation set. Figure~\ref{fig:psnr_curve} shows the per-frame PSNR for the different KL annealing strategies.

\begin{figure}[H]
    \centering
    \begin{tikzpicture}
    \begin{groupplot}[
            group style={
                    group size=1 by 2,
                },
            width=0.98\linewidth,
            height=0.3\linewidth,
            xlabel={Frame},
            grid=major,
            legend columns=7,
            legend to name=psnrlegend,
            legend style={
                    anchor=north,
                    column sep=1ex,
                    legend cell align=left,
                },
        ]
        \nextgroupplot[
            ylabel={PSNR (dB)},
            title={PSNR per Frame},
            tick align=outside,
        ]
        \addplot[red] table [
                col sep=comma,
                x=Step,
                y=Value
            ]{../csvs/tfr1_psnr.csv};
        \addlegendentry{TFR1}
        \addplot[orange] table [
                col sep=comma,
                x=Step,
                y=Value
            ]{../csvs/tfr2_psnr.csv};
        \addlegendentry{TFR2}
        \addplot[cyan] table [
                col sep=comma,
                x=Step,
                y=Value
            ]{../csvs/kl1_psnr.csv};
        \addlegendentry{KL1}
        \addplot[violet] table [
                col sep=comma,
                x=Step,
                y=Value
            ]{../csvs/kl2_psnr.csv};
        \addlegendentry{KL2}
        \addplot[brown] table [
                col sep=comma,
                x=Step,
                y=Value
            ]{../csvs/kl3_psnr.csv};
        \addlegendentry{KL3}
        \addplot[purple] table [
                col sep=comma,
                x=Step,
                y=Value
            ]{../csvs/kl4_psnr.csv};
        \addlegendentry{KL4}
    \end{groupplot}
\end{tikzpicture}
\ref*{psnrlegend}
\caption{PSNR per Frame in Validation Set.
    (a.) TFR1 is the configuration with \texttt{tfr = 1.0, tfr\_sde = 0, tfr\_d\_step = 0.1}, it has 28.97 dB average PSNR.
    (b.) TFR2 is the configuration with \texttt{tfr = 0.5, tfr\_sde = 10, tfr\_d\_step = 0.05}, it has 36.56(?) dB average PSNR.
    (c.) TFR3 is the configuration with \texttt{tfr = 0.0}, it has XXX dB average PSNR.
    (d.) KL1 is the configuration with cyclical KL schedule, it has XXX dB average PSNR.
    (e.) KL2 is the configuration with linear KL schedule, it has 27.95 dB average PSNR.
    (f.) KL3 is the configuration with constant KL schedule, it has 30.83 dB average PSNR.
    (g.) KL4 is the configuration with cyclical KL schedule with \texttt{kl\_anneal\_ratio = 0.5}, it has XXX dB average PSNR.
}

    \label{fig:psnr_curve}
\end{figure}

The results showed that PSNR typically remained stable in the early frames but dropped significantly after around frame 400. The model trained with a monotonic KL schedule achieved the highest average PSNR in early frames, while the cyclical schedule produced more stable long-term quality.

\subsection{Other Training Strategy Analysis (Bonus)}

In addition to the main training configurations, I explored several auxiliary strategies:

\begin{itemize}
    \item \textbf{Data Augmentation:} I applied Gaussian blur and normalization to the input frames. However, applying augmentation only to images (not labels) resulted in a distribution mismatch that degraded performance.
    \item \textbf{Optimizers:} I compared \texttt{Adam} and \texttt{AdamW}. The latter improved regularization and resulted in lower validation loss.
    \item \textbf{Schedulers:} I tested \texttt{MultiStepLR} and \texttt{CosineAnnealing}, but found that the default schedule provided the most consistent results.
\end{itemize}

Overall, the best results were achieved using the \texttt{AdamW} optimizer, cyclical KL annealing, and moderate teacher forcing decay.
