\section{Implementation Details}
\label{sec:implementation}

This section provides a detailed explanation of our implementation, which is structured into three parts: Multi-Head Self-Attention, Masked Visual Token Modeling (MVTM), and Iterative Decoding for Inpainting.

\subsection{Multi-Head Self-Attention}
In the Multi-Head Self-Attention implementation, the input tensor is first linearly projected into query (Q), key (K), and value (V) vectors.
Each of these vectors is reshaped into multiple heads to allow parallel attention computation.
We employ scaled dot-product attention independently on each head, which calculates attention scores by the dot product of Q and K, scaled by the square root of the dimension of the keys.
After computing the softmax scores, they are applied to the values (V).
The outputs from different heads are concatenated and passed through a final linear layer to produce the attention output.

Specific implementation details and dimension calculations can be referred to in the provided source code.

\inputminted[firstline=4, lastline=35]{python}{../models/Transformer/modules/layers.py}

\subsection{Masked Visual Token Modeling (MVTM)}
The MVTM module begins by encoding input images into latent tokens using a pretrained VQGAN encoder.
We randomly mask a subset of these tokens based on a Bernoulli distribution, replacing masked tokens with a dedicated mask token identifier.
The masked sequence is then passed through a bidirectional Transformer, which predicts the original tokens at masked positions.
The training objective is to minimize the cross-entropy loss between the predicted tokens and the ground-truth latent tokens.

Specific details regarding mask token selection, masking ratios, and Transformer configurations are described explicitly in the source code implementation.
\inputminted[firstline=33, lastline=35]{python}{../models/VQGAN_Transformer.py}
\inputminted[firstline=58, lastline=69]{python}{../models/VQGAN_Transformer.py}

\subsection{Iterative Decoding for Inpainting}

During inference, iterative decoding reconstructs masked regions of latent tokens through successive refinement.
Initially, all masked positions are filled with the mask token.
At each decoding iteration, the Transformer predicts token probabilities, and tokens with the highest confidence scores are fixed for subsequent iterations, while lower confidence tokens are remasked.
Various mask scheduling strategies, including cosine, linear, and square, determine how the masking ratio decreases throughout the iterations.
This iterative decoding continues until no masked tokens remain.

Implementation details such as temperature annealing, Gumbel noise application for sampling tokens, and specific mask scheduling functions are available within the provided implementation.

\inputminted[firstline=32, lastline=92, highlightlines={47,60,62-67}]{python}{../inpainting.py}

\inputminted[firstline=71, lastline=109]{python}{../models/VQGAN_Transformer.py}
