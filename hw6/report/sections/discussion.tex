\section{Discussion}

\subsection{Default Configuration}

The results of the default configuration are shows in Figure \ref{fig:default}.
The default configuration is using four down and up-blocks, and keep the attention layers only at the lowest resolution (bottom) layers to reduce memory usage.
And the model is trained for 300 epochs with a batch size of 32 and time steps of 1000.

\begin{figure}[h]
    \centering
    \begin{subfigure}{0.48\textwidth}
        \centering
        \includegraphics[width=\textwidth]{figures/default_test.png}
        \caption{Default Configuration Results on Test Set with accuracy of 0.9010}
        \label{fig:default_test}
    \end{subfigure}
    \hfill
    \begin{subfigure}{0.48\textwidth}
        \centering
        \includegraphics[width=\textwidth]{figures/default_new_test.png}
        \caption{Default Configuration Results on New Test Set with accuracy of 0.9271}
        \label{fig:default_new_test}
    \end{subfigure}
    \caption{Default Configuration Results and Accuracy}
    \label{fig:default}
\end{figure}

And the results of the manual test with ``red sphere'', ``cyan cylinder'', and ``cyan cube'' are shown in Figure \ref{fig:default_manual}.
\begin{figure}[h]
    \centering
    \includegraphics[width=\textwidth]{figures/default_manual_test.png}
    \caption{Default Configuration Results on Manual Test Set with accuracy of 1.0000}
    \label{fig:default_manual}
\end{figure}

And the loss curve of the default configuration is shown in Figure \ref{fig:loss_curve}.

\subsection{Same model with different time steps}

The results of the same model with different time steps are shown in Figure \ref{fig:step_100}.
The model is trained for 300 epochs with a batch size of 32 and time steps of 100.

\begin{figure}[h]
    \centering
    \begin{subfigure}{0.48\textwidth}
        \centering
        \includegraphics[width=\textwidth]{figures/step_100_test.png}
        \caption{100 Time Steps Results on Test Set with accuracy of 0.8802}
        \label{fig:step_100_test}
    \end{subfigure}
    \hfill
    \begin{subfigure}{0.48\textwidth}
        \centering
        \includegraphics[width=\textwidth]{figures/step_100_new_test.png}
        \caption{100 Time Steps Results on New Test Set with accuracy of 0.8906}
        \label{fig:step_100_new_test}
    \end{subfigure}
    \caption{100 Time Steps Results and Accuracy}
    \label{fig:step_100}
\end{figure}

And the results of the manual test with ``red sphere'', ``cyan cylinder'', and ``cyan cube'' are shown in Figure \ref{fig:step_100_manual}.
\begin{figure}[h]
    \centering
    \includegraphics[width=\textwidth]{figures/step_100_manual_test.png}
    \caption{100 Time Steps Results on Manual Test Set with accuracy of 1.0000}
    \label{fig:step_100_manual}
\end{figure}

And the loss curve of the 100 time steps is shown in Figure \ref{fig:loss_curve}.

\subsection{Discussion of Results}
The results of the default configuration and the same model with different time steps are shown in Figure \ref{fig:loss_curve}.
We can see that the loss is much larger than the default configuration, and the accuracy is also lower.
This means that it is harder to train the model with less time steps.


\begin{figure}[h]
    \begin{tikzpicture}
        \begin{axis}[
                title={Loss Curve of Default Configuration},
                xlabel={Epoch},
                xmin=0,
                xmax=300,
                ymode=log,
                ylabel={Loss},
                grid=major,
                width=\textwidth,
                height=0.5\textwidth,
                cycle list name=color list,
            ]
            \addplot table[col sep=comma, x=Step, y=Value] {csvs/default.csv};
            \addlegendentry{Train Loss with 1000 Time Steps}
            \addplot table[col sep=comma, x=Step, y=Value] {csvs/step_100.csv};
            \addlegendentry{Train Loss with 100 Time Steps}
        \end{axis}
    \end{tikzpicture}
    \caption{Loss Curve of Different Time Steps}
    \label{fig:loss_curve}
\end{figure}
