\subsection{Screenshots}

Results of the experiments are shown in the screenshots below:

\begin{figure}[H]
    \centering
    \begin{subfigure}{0.5\textwidth}
        \centering
        \includegraphics[width=\textwidth]{../results/hidden_16_lr_0.01_epochs_2000_BCELoss_sigmoid_linear/final.png}
        \caption{Dataset: Linear (Accuracy: 100\%)}
    \end{subfigure}%
    \begin{subfigure}{0.5\textwidth}
        \centering
        \includegraphics[width=\textwidth]{../results/hidden_16_lr_0.01_epochs_2000_BCELoss_sigmoid_xor/final.png}
        \caption{Dataset: XOR (Accuracy: 100\%)}
    \end{subfigure}
    \caption{Results of the experiments}
\end{figure}

The above experiments are conducted with the following hyperparameters:

\begin{itemize}
    \item Hidden units: 16
    \item Learning rate: 0.01
    \item Number of epochs: 2000
    \item Activation function: Sigmoid
    \item Loss function: Binary Cross Entropy
    \item Dataset: Linear, XOR
\end{itemize}

\subsection{Prediction Accuracy}

As shown in the screenshots, the model achieved 100\% accuracy on both the linear and XOR datasets.

\subsection{Learning Curve}

The learning curve of the experiments is shown below (the hyperparameters are the same as above):

\begin{figure}[H]
    \centering
    \begin{subfigure}[b]{0.45\textwidth}
        \centering
        \includegraphics[width=\textwidth]{../results/hidden_16_lr_0.01_epochs_2000_BCELoss_sigmoid_linear/loss.png}
        \caption{Dataset: Linear}
    \end{subfigure}
    \begin{subfigure}[b]{0.45\textwidth}
        \centering
        \includegraphics[width=\textwidth]{../results/hidden_16_lr_0.01_epochs_2000_BCELoss_sigmoid_xor/loss.png}
        \caption{Dataset: XOR}
    \end{subfigure}
    \caption{Learning curve of the experiments. The x-axis represents the number of epochs, and the y-axis represents the loss.}
\end{figure}
